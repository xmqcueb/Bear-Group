\documentclass{article}
  \title{The Relationship between the Development of Electric Power Industry and Carbon Dioxide Emission in China}
\author{Yueming Chen\\Zixuan Li\\Muqing Xiong\\Zixuan Li}
\begin{document}
  \maketitle
  \section{Introduction} With the rapid economic development and industrialization of China, how to realize the coordinated improvement of environment and economy has been a frequent topic of discussion. The increasing concentration of carbon dioxide is caused by the rapid development of the power industry, which is also a main reason of global warming. Actually, it’s hard to find a quantitative research about the correlation analysis between power industries’ progression and carbon dioxide’s emission. But China, as a country with a large population and a huge land area, is a main consumer and provider of fossil fuel. Therefore, energy saving and emission reduction has a great theoretical and realistic significance.
In this paper, on the basis of reviewing the research status quo at home and abroad, we take the emission of carbon dioxide, the generating average hours and the consumption of coal, oil and gas as sample. Using the linear regression method, we will have a thorough research on the relationship between the developmental level of Chinese power industry and different energy consumption levels.
  \section{Literature Review} (1) Domestic and Foreign Research on how the Development of Electric Power Industry (Technological Progress) Impact on the emission of Carbon Dioxide.According to the domestic and foreign researches, there are two main categories of the current application of the new technology generating fossil energy power: one is the most efficient type, like IGCC; another one is the Carbon reduction technology, like CCS. It can reduce more than 80 percent carbon emissions of the thermal power plant. CCS is an internationally recognized technology which can make a significant contribution to the reduction of carbon dioxide in the future. With these new technologies, we can improve the efficiency of fossil energy power and reduce the emission of carbon dioxide at the same time. We have concentrated on the innovation of fossil energy power generation to reduce carbon dioxide emissions more efficiently. Controlling and improving the quality of fuel is the technology most commonly used.
\\(2) Domestic and Foreign Research on the impact of the environmental condition on economic development.
How to achieve sustainable development has been a hot topic for a long time. Some domestic experts came up with some solutions. They believe that environmental conservation has a positive contribution to economic development. To strengthen environmental protection, our government need to invest a lot which can stimulate domestic demand, expand employment and promote economic growth. Some foreign researchers argue that high-tech science is the key to solve the contraction of environmental problem and economic development. Imagine that if there is a machine which can separate all the pollution from the fossil fuels, then we can be free from pollution. Both domestic and foreign experts have noticed the importance of environmental protection, so we should take the essence, go to the dregs and find the balance point between environmental protection and economic development.
  \section{Methodology}
      \subsection{ Data Selection} In this paper, we have studied carbon dioxide emissions from the power sector. The carbon dioxide emission is the main indicator of the greenhouse gas emission, which is also the air quality measurement to some extent. The electric power industry involves different sources of energy. And the aggregate consumption of coal, oil and natural gas accounted for 90 percent of the total energy consumption in the whole industry, so the study chooses the consumption of coal, oil and natural gas as variables. Besides, the generating average hour is an index measuring power generation efficiency of power plant, which reflects the electric power industry’s development. In conclusion, the study chooses carbon dioxide emissions, the consumption of coal, oil and natural gas and the generating average hours as variables.
      \subsection{ Data Calculation} Sources: (1) transformation coefficient of standard coal consumption derived from the National Development and Reform Commission Energy Research Institute. The scenario analysis of Chinese sustainable development of energy and carbon emissions [R] .2003; (2) transformation coefficient of carbon dioxide derived from intergovernmental climate change Committee.
\\Carbon emission = energy consumption×transformation coefficient of standard coal consumption ×transformation coefficient of carbon dioxide
\\(1)Measure initial data
The paper chooses eight kinds of fossil fuels including raw coal, coke, crude oil,	gasoline,	diesel,	fuel oil	and natural gas as the initial data, which influence the development of electric power industry. (The unit of raw coal, coke, crude oil,	gasoline,	diesel and	fuel oil is ton.	And the unit of natural gas is cubic metre.)
\\(2)Convert raw energy consumption into standard coal consumption
Multiplying the initial consumption of these kinds of fossil fuels by transformation coefficient of standard coal consumption, we can get the standard coal consumption.
\\(3)Calculate carbon emission.
Multiplying the standard coal consumption by transformation coefficient of carbon dioxide, we can calculate the carbon emission.
  \section{Stationarity test} We use the ADF test to examine the stationarity of our data. As we can see from the form,  the assumption of existing unit roots of the original series cannot be rejected at all 10\%, 5\%, 1\% significance level. After taking the first difference, there are still non-stationary series. Then taking the second order difference, we find that all variables reject the assumption of existing unit roots at 1\% and 5\% significance level. It shows that the second order difference of these series are steady.The carbon dioxide emission(E), the Coal consumption of the power industry(C) and the average number of hours using in power generation(H) are steady after taking the second order difference.
  \section{Models} According to the former results, we decided to estimate these three models:LNDE is the first order difference of the carbon dioxide emission(in logs). LNDH is the first order difference of the the average number of hours using in power generation(in logs). LNDC is the first order difference of the coal consumption(in logs). a、c、e are the coefficients. b、d、f are the intercepts
  $$LNDE=aLNDH+b$$
  $$LNDE=cLNDC+d$$
  $$LNDH=eLNDC+f$$
  \section{Results}In this paper, on the basis of reviewing the research status quo at home and abroad, taking the emission of carbon dioxide, the generating average hours and the consumption of coal, oil and gas as sample, using econometric and linear regression method, we had a thorough research on the relationship between China's development of power industry and the consumption of coal、oil、gas, the main results are as follows:
  \\(1)The result of the unit root hypothesis test  shows that LNE, LNC, LNH   are all integrated at order two, which is the former condition of cointegration analysis.
  \\(2) Model 1,2,3 are significant at level 5\%, 1\%, 5\%. It means the variables of model 2 may have the strongest linear relationship. That is the carbon dioxide emission may have relationship with the coal consumption.
 
\bibliographystyle{unsrt} 
\begin{thebibliography}{10}
\bibitem{1}China statistical yearbook [EB/OL], http://www.stats.gov.cn/tjsj/ndsj/
\bibitem{2}Chinese electricity council. Power industry statistics compiled [R]. 2006 - 2014.
\bibitem{3}The national electric power production investment statistics letters [J]. Journal of China power, 2002 01:91-92.
\bibitem{4}Electric power production annals of China electric power, 2003, 11:84-85.
\bibitem{5}Helian. "barometer" of supply and demand: thermal power 5300 hours [J]. Electric power enterprise management in China, 2011, 11:25 to 26.
\bibitem{6}For a rainy day, response to changes of power generation equipment use hours  [J]. Chinese electric power enterprise management, 2011,09:39-42.
\bibitem{7}Chinese electric power production and construction work review in 2000 and 2001, the main objectives and priorities (English) [J]. Journal of Electricity, 2001,01:3-6.
\bibitem{8}Pan Dai, Jiayong Zhou, Jie Tian, Tian Liu, Hao Zhou. Carbon reduction and  comprehensive optimization of Chinese electric power industry  [J]. Automation of electric power systems, 2013, 14:1-6.
\bibitem{9}Ding Haijing. Chinese electric power industry, such as the construction of climate index and fluctuation analysis [D]. Anhui university, 2014.
\bibitem{10}FanXing. Analysis of Chinese carbon emissions measurement and research about reduction methods [D]. Liaoning university, 2013.
\end{thebibliography}
\end{document} 
